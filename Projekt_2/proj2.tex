\documentclass[twocolumn, 11pt, a4paper]{article}[16.03.2023]
    \usepackage[left=1.4cm, top=2.3cm, text={18.2cm, 25.2cm}]{geometry}
    \usepackage[utf8]{inputenc}
    \usepackage[czech]{babel}
    \usepackage[IL2]{fontenc}
    \usepackage{amsthm, amssymb, amsmath}
    \usepackage{times}
\newtheorem{definition}{Definice}
\newtheorem{sentence}{Věta}


\begin{document}

\begin{titlepage}
    \begin{center}
        
        {\Huge \textsc{Vysoké učení technické v Brně}\\[0.5em]}
        {\huge \textsc{Fakulta informačních technologií}}
        \vspace{\stretch{0.382}}
    
        {\LARGE Typografie a publikování -- 2.projekt\\[0.4em]
        Sazba dokumentů a matematických výrazů}
    
        \vspace{\stretch{0.618}}
        {\Large \the\year \hfill Roman Machala (xmacha86)}
    \end{center}
    
\end{titlepage}

\section*{Úvod}

V této úloze si vyzkoušíme sazbu titulní strany, 
matematických vzorců, prostředí a dalších textových struktur 
obvyklých pro technicky zaměřené texty\,--\,například Definice~\ref{Definice} nebo rovnice~\eqref{rovnice3} na straně~\pageref{1}. 
Pro vytvoření těchto odkazů používáme kombinace příkazů \verb|\label|, \verb|\ref|, \verb|\eqref| a \verb|\pageref|. 
Před odkazy patří nezlomitelná mezera. Pro zvýrazňování textu jsou zde 
několikrát použity příkazy \verb|\verb| a \verb|\emph|. 

Na titulní straně je použito prostředí titlepage a sázení nadpisu 
podle optického středu s využitím \emph{přesného} zlatého řezu. 
Tento postup byl probírán na přednášce. 
Dále jsou na titulní straně použity čtyři různé velikosti písma a mezi 
dvojicemi řádků textu je použito odřádkování se zadanou relativní 
velikostí 0,5\,em a 0,4\,em\footnote[1]{Nezapomeňte použít správný typ mezery mezi číslem a jednotkou.}.

\section{Matematický text}
V této sekci se podíváme na sázení matematických symbolů a výrazů v plynulém 
textu pomocí prostředí math. 
Definice a věty sázíme pomocí příkazu \verb|\newtheorem| s využitím balíku amsthm. 
Někdy je vhodné použít konstrukci \verb|${}$| nebo \verb|\mbox{}|, která říká, že (matematický) text 
nemá být zalomen.
\begin{definition}
    \emph{Zásobníkový automat} (ZA) je definován jako sedmice tvaru A = $\left(Q, \Sigma, \Gamma, \delta, q_0, Z_0, F\right)$, kde: \label{Definice}
\end{definition}
        \begin{itemize} 
            \item $Q$ je konečná množina \emph{vnitřních (řídících) satvů,}
            \item $\Sigma$ je konečná \emph{vstupní abeceda,}
            \item $\Gamma$ je konečná \emph{zásobníková abeceda,}
            \item $\delta$ je \emph{přechodová funkce} $Q \times (\Sigma\,\cup\,\{\epsilon\})\times\Gamma\rightarrow2^{Q\times\Gamma^\ast},$
            \item $q_0 \in Q$ je \emph{počáteční stav,} $Z_0 \in \Gamma$ je \emph{startovací symbol zásobníku} a $F \subseteq Q$ je množina \emph{koncových stavů.} 
        \end{itemize}

     \emph{Nechť} $P=(Q,\Sigma,\Gamma,\delta,q_0,Z_0,F)$ \emph{je ZA.} konfigurací \emph{nazveme
        trojici} $(q,\omega,\alpha) \in Q\times\Sigma^{\ast}\times\Gamma^{\ast}$, \emph{kde} q\emph{je aktuální 
        stav vnitřního řízení}, $w$ \emph{je dosud nezpracovaná část vstupního řetězce a} $\alpha=Z_{i_1}Z_{i_2}\dots Z_{i_k}$ \emph{je obsah zásobníku}.

    \subsection{Podsekce obsahující definici a větu}
    \begin{definition}
        \emph{Řetězec $\omega$ and abecedou $\Sigma$ je přijat ZA} $\mathnormal{A}$~jestliže $(q_0,\omega,Z_0) \underset{A}{\overset{\ast}{\vdash}} (q_F,\epsilon,\gamma)$ pro nějaké $\gamma\in\Gamma^\ast$ a $q_F\in F$.
        Množina $L(A)=\{w\ |\ w$ je přijat ZA $\mathnormal{A}\} \subseteq \Sigma^\ast$ je \emph{jazyk přijímaný za ZA} $\mathnormal{A}$.
    \end{definition}
    \begin{sentence}
        Třída jazuků, které jsou přijímány ZA, odpovídá \emph{bezkontextovým jazykům}.
    \end{sentence}

\section{Rovnice}
    Složitější matematické formulace sázíme mimo plynulý text pomocí prostředí \verb|displaymath|.
    Lze umístit i několik výrazů na jeden řádek, ale pak je třeba tyto vhodně oddělit, například příkazem \verb|\quad|.

        \begin{displaymath}
            1^{2^{3}} \ne \Delta_{\Delta_{\Delta^3}^2}^1 \quad y^{11}_{22}-\sqrt[9]{x+\sqrt[7]{y}} \quad x > y_1 \leq y^2
        \end{displaymath}
    V rovnici~\eqref{rovnice2} jsou využity tři typy závorek s různou \emph{explicitně} definovanou velikostí.
    Také nepřehlédněte, že následující tři rovnice mají zarovnaná rovnítka, a použijte k~tomuto účelu vhodné prostředí.

\begin{align}
        -\cos^{2} \beta \;&=\;\frac{\frac{\frac{1}{x}+\frac{1}{3}}{y}+1000}{\underset{j=2}{\overset{8}{\Pi\,q_j}}} \\
        \biggl(\Bigl\{ b\,\star \bigl[ 3\div 4\bigr] \Bigr\}^{\frac{2}{3}}\biggr)\,&=\,\log_{10}{x} \label{rovnice2}\\
        \int_{a}^{b}f\!\left(x\right)dx \,&=\, \int_{c}^{d}f\!\left(y\right)dy \label{rovnice3}
\end{align}

V této větě vidíme, jak vypadá implicitní vzsázení limity $\lim_{m\rightarrow\infty} f(m)$ v normálním odstavci textu.
Podobně je to i s dalšími symboly jako $\bigcup_{N\in\mathcal{M}} N$, či $\sum_{i=1}^{m} x^2_i$. 
S vynucením méně úsporné sazby příkazem \verb|\limits| budou vzorce vysázeny v podobě
$\lim\limits_{m\rightarrow\infty}f(m)$ a $\sum\limits_{i=1}^{m}x^4_i$.

\section{Matice}
Pro sázení matic se velmi často používá prostředí \verb|array| a závorky (\verb|\left|,\verb|\right|).
$$B=\left|
\begin{array}{cccc}
    b_{11} & b_{12} & \dots & b_{1n}\\
    b_{21} & b_{22} & \dots & b_{2n}\\
    \vdots & \vdots & \ddots & \vdots\\
    b_{m1} & b_{m2} & \dots & b_{mn}
\end{array}
\right| = \left|
\begin{array}{cc}
    t & u\\
    v & w
\end{array}
\right| = tw - uv$$ 

$$\mathbb{X}= \mathbf{Y} \Longleftrightarrow \left[
    \begin{array}{ccc}
        & \Omega+\Delta & \hat{\psi}\\
        \vec{\pi} & \omega & 
    \end{array}
\right] \ne 42$$

Prostředí \verb|array| lze úspěšně využít i jinde, například 
na pravé straně následující rovnice. Kombinační číslo na levé straně vysázejte pomocí příklazu \verb|\binom|.

$$\binom{n}{k} = 
\left\{
\begin{array}{c l}
0 & \emph{pro}\,k < 0 \\
\frac{n!}{k!(n-k)!} &  \emph{pro}\,0\le k \le n \\
0 & \emph{pro}\,k > 0
\end{array}
\right. $$
\label{1}

\end{document}
