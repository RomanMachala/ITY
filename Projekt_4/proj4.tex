\documentclass[11pt, a4paper]{article}[14.04.2023]
    \usepackage[left=2cm, top=3cm, text={17cm, 24cm}]{geometry}
    \usepackage[utf8]{inputenc}
    \usepackage[czech]{babel}
    \usepackage[IL2]{fontenc}
    \usepackage{times}
    

\begin{document}

    \begin{titlepage}
        \begin{center}
            {\Huge \textsc{Vysoké učení technické v Brně}\\[0.5em]}
            {\huge \textsc{Fakuklta informačních technologií}}
            \vspace{\stretch{0.382}}

            {\LARGE Typografie a publikování -- 4.projekt\\[0.4em]}
            {\huge Bibliografické citace}

            \vspace{\stretch{0.618}}
            {\Large \today \hfill Roman Machala}

        \end{center}
    \end{titlepage}

\section{Úvod}


Pod pojmem typografie a publikování si lidé neorientující se v tomto oboru představí například psaní článků a vydávání novin a časopisů.
Ovšem skrývá se pod ním mnohem více, než si člověk dokáže představit.

\section{Typografie a publikace obecně}


S vývinem člověka jako takového přišla i řeč a písmo. S písmem se objevil nový druh komunikace mezi lidmi, díky kterému jsme schopni uchovávat naše myšlenky a znalosti po celá staletí.
Nemá smysl tady zacházet do úplných detailů, pojďme si proto říci nějaké nejdůležitější věci týkající se typografie.
Typografie se nezabývá pouze historií, ale taky jak by měly dané dokumenty vypadat \cite{font}. Řeší, pro jakou skupinu jsou určeny a tomu se pak snaží přizpůsobovat jednotlivé požadavky na daný dokument.
Návodů jak vytvořit ten správný text je na internetu mnoho, mně se ovšem zalíbil trošku detailnější postup zde \cite{rybicka2}.

\section{Prostředí pro tvorbu dokumentů}


První prostředí pro tvorbu dokumentů, které snad kohokoliv napadne je \textit{Microsoft Word}. Je to poměrně rozšířený a oblíbený nástroj ale ani zdaleka není jediný.
Hodí se pro mnoho dokumentů, ale se zvyšující se náročností daného zadání mi přijde méně vhodnější. Pro náročněšjší sázení jsou vhodnější profesionální prostředí jako je například \LaTeX.
Tyto dvě zmíněné prostředí bychom mohli porovnávat donekonečna a jejich výhody a nevýhody již byly několikrát popsány. Ve výsledku to samozřejmě záleží na osobních preferencích autora \cite{baka}.

\section{Prostředí \LaTeX}


Prostředí \LaTeX\;je jedním z nejrozšířenějších prostředí pro tvorbu dokumentů. Poskytuje nám řadu nových možností a funkcionalit.
Může být ovšem trošku náročnější na pochopení a na naučení se jej plynule používat. Obávat se toho ovšem nikdo nemusí, jelikož návodů, jak správně uchopit toto prostředí je mnoho.
Nejrozšířenější a nejoblíbenější, především mezi studenty Českých výsokých škol je pak \cite{rybicka1}.

Toto prostředí, jak již bylo dříve zmíněno je především vhodné pro tvorbu složitějších textů.
Hlavním problémem ze subjektivního hlediska je tvorba tabulek. V prostředí \textit{Microsoft Word} lze jednoduše vložit tabulky i z prostředí \textit{Microsoft Excel}, kdežto v \LaTeX{u}
\;je mnohdy boj vytvořit i jednoduchou tabulku. Na druhou stranu \LaTeX vyniká především v sázení matematických, fyzikálních a chemických vzorců, kde \textit{Microsoft Word} může trošku zaostávat.
Další přednosti prostředí \LaTeX{u} jsou popsány např. zde \cite{diplomka}.

\section{\LaTeX\;a studenti}

Již jednou bylo zmíněné, že \LaTeX\;je oblíbený i mezi studenty. Je tomu především u studentů se zaměřením na 
nějaké technické obory, ovšem nemusí tomu tak být pořád.
Studenti jej využívají pro tvorbu technických zpráv, různých protokolů ale především pro tvorbu bakalářských a diplomových prací. 
Největším kámenem úrazu, nejen u studentů ale v typografii obecně, jsou citace ostatních autorů a dodržování autorského práva, tedy rozlišit, co bylo převzato a co ne.
Skvělým návodem pro nejen citaci obecně, ale i pro citaci v prostředí \LaTeX\;naleznete zde \cite{davidekm}.
\\
\begin{quotation}
    Psaní dokumentů podléhá mnoha typografickým normám, tj. nemůžete si psát dokumenty, jak se vám zlíbí. \cite{Jirkovystranky} 
\end{quotation}

\newpage

\bibliographystyle{czechiso}
\renewcommand{\refname}{Literatura}
\bibliography{citace}
\end{document}
     
    

    
